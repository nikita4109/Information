\documentclass{resume}
\usepackage[hidelinks]{hyperref}
\usepackage{lipsum}

\name{Nikita Samsonov}
\contact{Israel}
\contact{+972584832213}
\contact{\href{mailto:nikita4109@gmail.com}{nikita4109@gmail.com}}
\contact{\href{https://github.com/nikita4109}{github.com/nikita4109}}
\contact{\href{https://www.linkedin.com/in/nikita4109/}{linkedin.com/in/nikita4109}}
 
\begin{document}
\makeheader

\begin{ResumeSection}{skills}
    \newcommand{\skill}[2]{\textbf{#1} - #2}
    \begin{ResumeSubsection}{org=Engineering}
        \begin{itemize}
            \item \skill{Software Development}{Go, C/C++, PostgreSQL, Git, Docker, Graphana, EVM}
            \item \skill{Data Engineering}{Python, numpy, matplotlib, pandas}
            \item \skill{Text Processing}{LaTeX}
        \end{itemize}
    \end{ResumeSubsection}

    \begin{ResumeSubsection}{org=General}
        \begin{itemize}
            \item \skill{Languages}{English -- Upper-Intermediate, Russian -- Native, Hebrew -- Beginner}
        \end{itemize}
    \end{ResumeSubsection}
\end{ResumeSection}

\begin{ResumeSection}{experience}
    \begin{ResumeSubsection}{org=\href{https://www.jetbrains.com/}{JetBrains},position={Software Engineering Intern},duration=November 2021 – April 2022}
        \begin{itemize}
            \item {
                Extended the functionality of the ReSharper by adding the ability to choose the desired configuration for indentation and alignment in a column without affecting the source code.
            }

            \item {
                Added a settings section for virtual alignment similar to the settings for normal alignment.
            }

            \item {
                Fixed a bug that aligned the column not by user code, but by the visual element.
            }

            \item {
                Wrote tests for implemented classes and some integration tests to check the interactions of new code and the existing code.
            }
        \end{itemize}
    \end{ResumeSubsection}

    \begin{ResumeSubsection}{org=Yandex,position={Software Engineering Intern},duration=July 2022 – October 2022}
        \begin{itemize}
            \item {
                Implemented a search functionality for an Amazon S3 bucket using PostgreSQL.
            }

            \item {
                Implemented the ability to subscribe to an event instead of polling the API every second, which helped offload the API and speed up the service.
            }

            \item {
                Accelerated and simplified the contribs update process by automating it. $95\%$ of contribs were updated automatically and only $5\%$ had to resolve conflicts and change source code manually.
            }

            \item {
                For all the pieces of code I wrote, I tracked metrics. This was necessary to analyze how quickly and correctly my part of the service worked.
            }

            \item {
                Wrote unit tests for implemented modules. Tested multi-threaded code for data races and deadlocks.
            }

            \item {
                Used Go, C++, Docker, local message broker and PostgreSQL.
            }
        \end{itemize}
    \end{ResumeSubsection}

    \begin{ResumeSubsection}{org=MidGPTBot,position={Software Engineering},duration=January 2023 – March 2023}
        \begin{itemize}
            \item {
                Paralleling each user so that server resources are used as optimally as possible and the user gets the best possible experience using the service.
            }

            \item {
                Optimising the cost of using neural network APIs.
            }

            \item {
                Aggregation of different neural network APIs and improvement of their response \- quality by means of prompts.
            }

            \item {
                Deployment and management in the cloud. Deploying bot using Docker and Google Cloud Platforms.
            }

            \item {
                Visualization of collected metrics using Graphana.
            }

            \item {
                Used Go, Docker, Graphana, PostgreSQL and Google Cloud Platforms.
            }
        \end{itemize}
    \end{ResumeSubsection}
    \begin{ResumeSubsection}{org=Cryptocurrency arbitrage,position={Software Engineering},duration=March 2023 – June 2023}
        \begin{itemize}
            \item {
                Created an infrastructure that allowed to find arbitrage opportunities for given pairs of dex and token.
            }

            \item {
                Created infrastructure for token exchange and bridging between chains.
            }

            \item {
                Found bots that are engaged in arbitrage and conducted their analytics, identified tokens and dexes that they use for arbitrage, found their weaknesses.
            }

            \item {
                Optimized arbitrage, found patterns where bridging is faster, found the shortest ways to exchange tokens.
            }

            \item {
                Used Go, Docker, Solidity, EVM.
            }
        \end{itemize}
    \end{ResumeSubsection}
\end{ResumeSection}

\begin{ResumeSection}{education}
    \begin{ResumeSubsection}{org={Higher School of Economics},duration={Sep 2020 - Jun 2022},position={Bachelor in Software Engineering}}
        \\\bf GPA: 7.93 
    \end{ResumeSubsection}
\end{ResumeSection}

\begin{ResumeSection}{ACHIEVEMENTS}
    \begin{itemize}
        \item Rating on \href{https://codeforces.com/profile/nikita4109}{Codeforces}: 1911
        \item Prize-winner at the All-Siberian Open Olympiad Informatics. 
        \item Prize-winner at Lomonosov olympiad in Computer Science and Programming.
    \end{itemize}
\end{ResumeSection}

\end{document}
